%-*- mode: latex ;mode: visual-line -*-

% A formula sheet for:
%   University of Waterloo/Guelph Phys 701/7010, Quantum Mechanics I

% For history, pull requests etc..., see github repo:
%   https://github.com/jddmartin/university_of_waterloo_phys701_formula_sheet

% Abbreviations used in references:
% SN: Sakurai and Napolitano, Quantum Mechanics, 2nd ed.
% SAQM: Schwabl, Advanced quantum mechanics, 4th ed.
% SHMT5: Spiegel, Schaum's outline of math formula ..., 5th ed.

\documentclass[11pt]{article}
\usepackage[left=1.5cm,top=2.0cm,right=2.25cm,bottom=3.0cm]{geometry}
\usepackage{siunitx}
\usepackage{tikz}
\usepackage{pdfpages}
\usepackage{multicol}
\usepackage{braket}
\usepackage{amsmath}
\usepackage[colorlinks=true,linkcolor=blue,urlcolor=blue]{hyperref}

% for the following see: https://tex.stackexchange.com/a/11144
\newcommand{\Lagr}{\mathcal{L}}
\newcommand{\CurD}{\mathcal{D}}
\newcommand{\CurJ}{\mathcal{J}}

\newcommand{\vect}[1]{\boldsymbol{\mathbf{#1}}}

\newcommand{\lambdabar}{{\mkern0.75mu\mathchar '26\mkern -9.75mu\lambda}}

\setlength{\parskip}{5pt}
\setlength{\parindent}{0pt}
\setlength{\columnseprule}{0.4pt}
\setlength{\columnsep}{1in}

\begin{document}

\begin{center}
{\large University of Waterloo/Guelph, Phys 701/7010, Quantum Mechanics I, Formula sheet}
\end{center}

\begin{multicols}{2}

{\bf Basic Formulae}

\begin{equation} % Reference: SN1.4.53
\braket{(\Delta A)^2} \braket{(\Delta B)^2}
  \ge \frac{1}{4} |\!\braket{[A,B]}|^2
\end{equation}

\begin{equation}
\Delta A \equiv A - \braket{A}
\end{equation}

\begin{equation}
\CurJ (\Delta \vect{x}') = \exp \left (-\frac{i}{\hbar} \Delta \vect{x}' \cdot \vect{p} \right)
\end{equation}

\begin{equation}
[\,x,p\,] = i\hbar
\end{equation}

\begin{equation}
\vect{p} = -i \hbar \vect{\nabla}
\end{equation}

\begin{equation}
\braket{x'|p'} = \frac{1}{\sqrt{2\pi\hbar}} \exp (i\frac{p'x'}{\hbar})
\end{equation}

\begin{equation}
i\hbar \frac{\partial \psi}{\partial t} = (-\frac{\hbar^2}{2m} \nabla^2 + V) \psi
\end{equation}

\begin{equation}
\psi = \frac{u(r)}{r} Y_{\ell,m}(\theta,\phi)
\end{equation}

\begin{equation}
\left(
  -\frac{\hbar^2}{2m_r} \frac{d^2}{d r^2}
  + \frac{\hbar^2}{2m_r} \frac{\ell (\ell+1)}{r^2}
  + V(r)
\right) u(r) = E u(r)
\end{equation}
where $m_r = m_1 m_2 / (m_1 + m_2)$.

\begin{equation}
\vect{\nabla} \cdot \vect{j} = - \frac{\partial \rho}{\partial t}
\end{equation}

\begin{equation}
\vect{j} = - \frac{i \hbar}{2m} \: [\psi^* \vect{\nabla} \psi - \psi \vect{\nabla} \psi^*]
\end{equation}

\begin{equation}
e^{A+B}=e^A \: e^B \: e^{-\frac{1}{2}[A,B]}
\end{equation}
{\footnotesize for \( [[A,B],A]=[[A,B],B]=0 \) .}

\columnbreak

{\bf Harmonic Oscillator}

\begin{equation}
H = \frac{p^2}{2m} + \frac{1}{2} m \omega^2 \: x^2
\end{equation}

\begin{equation}
a = \sqrt{\frac{m\omega}{2\hbar}} \left( x+\frac{ip}{m\omega} \right)
\end{equation}

\begin{equation}
a^{\dagger} = \sqrt{\frac{m\omega}{2\hbar}} \left( x-\frac{ip}{m\omega} \right)
\end{equation}

\begin{equation}
[\:a, a^{\dagger}] = 1
\end{equation}

\begin{equation}
a \ket{n} = \sqrt{n} \ket{n-1}
\end{equation}

\begin{equation}
a^{\dagger} \ket{n} = \sqrt{n+1} \ket{n+1}
\end{equation}

\begin{equation}
H = \hbar\omega(a^{\dagger}a+\frac{1}{2})
\end{equation}

\begin{equation}
E = \hbar\omega(n+\frac{1}{2})
\end{equation}
where $n=0,1,2,\ldots$.

{\bf Spinless non-relativistic hydrogen}

\begin{equation}
E = - \frac{mc^2}{2n^2} (Z\alpha)^2
\end{equation}
where $n=1,2,3,\ldots$.

{\bf Angular Momentum}

\begin{equation}
\CurD_{\vect{\hat{n}}} (\Delta \theta )
  = \exp \left(-\frac{i}{\hbar} \: \Delta \theta \:
               \vect{J} \cdot \vect{\hat{n}} \right )
\end{equation}

\begin{equation}
[J_a,J_b] = i \hbar \epsilon_{abc} J_c
\end{equation}

\begin{equation}
\vect{J}^2 \ket{j,m} = \hbar^2 j (j+1) \ket{j,m}
\end{equation}

\begin{equation}
J_z \ket{j,m} = \hbar m \ket{j,m}
\end{equation}

\begin{equation}
J_{\pm} = J_x \pm i J_y
\end{equation}

\begin{equation}
J_{\pm} \ket{j,m} = \hbar \sqrt{j(j+1)-m(m\pm1)} \ket{j,m\pm1}
\end{equation}

\begin{equation}
\ket{j_1j_2;jm} =
 \sum_{m_1,m_2} \braket{j_1j_2;m_1m_2|j_1j_2;jm} \ket{j_1j_2;m_1m_2}
\end{equation}

\begin{equation}
\braket{j_1j_2;m_1m_2|j_1j_2;jm}
 = \braket{j_1j_2;jm|j_1j_2;m_1m_2}
\end{equation}

\begin{equation}
\braket{j_1j_2;j_1,j-j_1|j_1j_2;jj} > 0
\end{equation}

\begin{equation}
\begin{split}
\sum_{m_1,m_2} \braket{j_1j_2;m_1m_2|j_1j_2,jm} \braket{j_1j_2;m_1m_2|j_1j_2,j'm'} \\ = \delta_{j,j'}\delta_{m,m'}
\end{split}
\end{equation}

\begin{equation}
\begin{split}
\sum_{j,m} \braket{j_1j_2;m_1m_2|j_1j_2,jm} \braket{j_1j_2;m_1'm_2'|j_1j_2,jm} \\ = \delta_{m_1,m_1'}\delta_{m_2,m_2'}
\end{split}
\end{equation}

\begin{equation}
\begin{split}
\bra{\alpha' j' m'}T^{(k)}_q\ket{\alpha j m} =
  \frac{\bra{\alpha ' j'}|T^{(k)}|\ket{\alpha j}}{\sqrt{2j+1}} \times \\
  \braket{jk;mq|jk;j'm'}
\end{split}
\end{equation}

\begin{equation}
\vect{L} = \vect{r} \times \vect{p}
\end{equation}

{\bf Spin-1/2}

\begin{equation}
\vect{J} = \frac{\hbar}{2} \vect{\sigma}
\end{equation}
where
\begin{equation}
\sigma_x = \left(
           \begin{array}{cc}
              0 & 1 \\
              1 & 0
           \end{array}
           \right)
\end{equation}

\begin{equation}
\sigma_y = \left(
           \begin{array}{cc}
              0 & -i \\
              i & 0
           \end{array}
           \right)
\end{equation}

\begin{equation}
\sigma_z = \left(
           \begin{array}{cc}
              1 & 0 \\
              0 & -1
           \end{array}
           \right)
\end{equation}

\begin{equation}
[ \sigma_i, \sigma_j ] = 2i \epsilon_{i,j,k} \sigma_k
\end{equation}

\begin{equation}
[\,\sigma_i,\sigma_j\,]_+ = 2 \delta_{i,j}
\end{equation}
\begin{equation}
e^{- \frac{i}{2} \, \vect{\sigma} \cdot \vect{\hat{n}} \, \phi
     }
  = \cos(\phi /2) - i \vect{\sigma} \cdot \vect{\hat{n}} \sin(\phi /2)
\end{equation}

{\bf EM Fields (rationalized MKSA units)}

\begin{equation}
\vect{B} = \vect{\nabla} \times \vect{A}
\end{equation}

\begin{equation}
\vect{E} = -\vect{\nabla} \phi - \frac{\partial \vect{A}}{\partial t}
\end{equation}

\begin{equation}
H = \frac{(\vect{p} - q \vect{A})^2}{2m} + q \phi
\end{equation}

\begin{equation}
\Lagr = \frac{1}{2} m \vect{v} \cdot \vect{v} + q \vect{v} \cdot \vect{A} - q \phi
\end{equation}

\begin{equation}
\vect{A} = - \frac{1}{2} \: (\vect{r} \times \vect{B_{\rm constant}})
\end{equation}

\begin{equation}
A^{i} \equiv (V/c,\vect{A})
\end{equation}

\begin{equation}
\vect{\mu}_{\rm magnetic} = \frac{q}{2m} \vect{\ell}
\end{equation}

\begin{equation}
\vect{\mu}_{\rm magnetic} = g \frac{q}{2m} \vect{j}
\end{equation}

\begin{equation}
U_{\rm magnetic} = - \vect{\mu}_{\rm magnetic} \cdot \vect{B}
\end{equation}

{\bf Scattering}

\begin{equation}
R = F \sigma
\end{equation}

% Ansatz wave-function
\begin{equation}
\psi = A \left [
         e^{ikz} + f(\theta,\phi) \frac{e^{ikr}}{r}
         \right ]
\end{equation}

\begin{equation}
\frac{d \sigma}{d \Omega} = |f(\theta,\phi)|^2
\end{equation}

\begin{equation}
f(\theta) = \frac{1}{k} \sum_{\ell=0}^{\infty}
            (2\ell+1) e^{i\delta_{\ell}} \sin(\delta_{\ell})
            P_{\ell}(\cos \theta)
\end{equation}

\begin{equation}
\sigma = \frac{4\pi}{k^2} \sum_{\ell = 0}^{\infty} (2\ell+1)
         \sin ^2(\delta_{\ell})
\end{equation}

\begin{equation}
u(r) \propto \sin \left( kr - \frac{1}{2}\ell\pi+\delta_{\ell}
                  \right)
\end{equation}

% Born approximation:
\begin{equation}
f(\theta,\phi) \approx - \frac{m}{2\pi\hbar^2}
                 \int e^{i(\vect{k'}-\vect{k}) \cdot \vect{r_0}} V(\vect{r}_0)
                      \: d^3 \vect{r}_0
\end{equation}
where $\vect{k'} = k \vect{\hat{z}}$.
\begin{equation}
f(\theta) \approx - \frac{2m}{\hbar^2 q}
          \int_{0}^{\infty} r V(r) \sin(q r) \: dr,
\end{equation}
where $q = 2 k \sin(\theta/2)$.

\begin{equation}
u(r) \propto (r - a_s)
\end{equation}

\begin{equation}
\sigma = 4\pi a_s^2
\end{equation}

\begin{equation}
E_{\rm bind} \approx \frac{\hbar^2}{2ma_s^2}
\end{equation}

\begin{equation}
  \frac{d\sigma}{d\Omega} = \frac{Z^2 q^4}{(4\pi\epsilon_0)^2}
  \frac{1}{16E^2 \sin^4 \theta/2} ,
\end{equation}

\begin{equation}
\frac{d\sigma}{d \Omega}=\left(\frac{\hbar c}{E}\right)^2 \quad \frac{Z_1^2 Z_2^2 \alpha^2}{16 \sin ^4 \theta / 2}
\end{equation}

\begin{equation}
  \sigma_{\ell} = \frac{4\pi}{k^2}
                  \frac{(2\ell+1)(\Gamma/2)^2}{[(E-E_r)^2+\Gamma^2/4]}
\end{equation}

\begin{equation}
\frac{d(\cot \delta_{\ell})}{dE} \bigg\rvert_{E=E_r} = - \frac{2}{\Gamma}
\end{equation}


{\bf Interaction picture}

\begin{equation} % Reference: SN Eq. 5.5.5
\ket{\alpha,t_0;t}_{I} =
  e^{iH_0t/\hbar} \ket{\alpha,t_0;t}_{S}
\end{equation}

\begin{equation} % Reference: SN Eq. 5.5.7
V_I = e^{iH_0t/\hbar}Ve^{-iH_0t/\hbar}
\end{equation}

\begin{equation} % Reference: SN Eq. 5.5.11
i\hbar \frac{\partial}{\partial t} \ket{\alpha,t_0;t} = V_I \ket{\alpha,t_0;t}
\end{equation}

\begin{equation} % Reference: SN Eq. 5.5.13
\ket{\alpha,t_0;t}_I = \sum_n c_n(t) \ket{n}
\end{equation}

\begin{equation} % Reference: SN Eq. 5.5.15
i\hbar \frac{d}{dt} c_n(t)  = \sum_m V_{nm} e^{i\omega_{nm}t}c_m(t)
\end{equation}

{\bf Time-dependent perturbation theory}

\begin{equation}
c_n(t) = c_n^{(0)} + c_n^{(1)} + c_n^{(2)} + \dots
\end{equation}

\begin{align}
c_n^{(0)}(t) & = \delta_{n,i} \\
c_n^{(1)}(t) & = \frac{-i}{\hbar}
  \int_{t_0}^t e^{i\omega_{n,i}t'} V_{n,i}(t') \: dt' \\
c_n^{(2)}(t) & = \left( \frac{-i}{\hbar} \right)^2
\sum_m \int_{t_0}^{t} dt' \int_{t_0}^{t'} dt''  \nonumber \\
&
e^{i\omega_{n,m}t'} V_{n,m}(t') e^{i\omega_{m,i}t''} V_{m,i}(t'')
\end{align}


\begin{equation}
|c_n^{(1)}|^2 =
  \frac{4|V_{ni}|^2}{|E_n-E_i|^2} \sin^2
    \left [
      \frac{(E_n-E_i)t}{2\hbar}
    \right ]
\end{equation}
{\footnotesize for $V$ switched on at $t=0$, but otherwise time-independent.}

\begin{equation}
w_{i\rightarrow [n]} =
\frac{2\pi}{\hbar} \overline{|V_{n,i}|^2} \rho(E_n)_{E_n \approx E_i}
\end{equation}

{\bf Relativistic one-particle wave equations}

\begin{equation}
E = \sqrt{(pc)^2+(mc^2)^2}
\end{equation}

\begin{equation}
\lambdabar = \frac{\hbar c}{mc^2}
\end{equation}

\begin{equation}
A^i \equiv (A^0, \vect{A})
\end{equation}

\begin{equation}
A_i \equiv (A^0, -\vect{A})
\end{equation}

\begin{equation}
A^i B_i = A_i B^i \:\: {\rm invariant}
\end{equation}

\begin{equation}
\partial_i \equiv (\frac{1}{c}\partial_t, \vect{\nabla})
\end{equation}

\begin{equation}
\partial^2 \equiv \partial_{i} \partial^{i}
\end{equation}

\begin{equation}
\left( \frac{\hbar^2}{c^2} \partial^2 + m^2 \right) \psi = 0
\end{equation}

\begin{equation}
i \hbar \frac{\partial \psi}{\partial t} =
 \left( c \: \vect{\alpha} \cdot \vect{p} + \beta mc^2 \right) \psi
\end{equation}

\begin{equation}
\vect{\alpha} = \left(
                   \begin{array}{cc}
                     0 & \vect{\sigma} \\
                     \vect{\sigma} & 0
                   \end{array}
                \right)
\:\:\:\:
\beta = \left(
           \begin{array}{cc}
              I & 0 \\
              0 & -I
           \end{array}
        \right)
\end{equation}

\begin{equation}
\vect{j} = c \: \psi^{\dagger} \vect{\alpha} \psi
\end{equation}

\begin{equation}
i \hbar \frac{\partial \psi}{\partial t} =
\left (-\frac{\hbar^2}{2m} \nabla^2
  - g \frac{q}{2m} \frac{\hbar}{2} \vect{\sigma}\cdot \vect{B} \right)
  \psi
\end{equation}

% Refence Eq. 8.2.37 of SAQM
\begin{align}
  E_{n,j} & = mc^2 \:\:\: \times \nonumber \\
    & \left[1 +
    \left(
    \frac{Z\alpha}{
      n - (j+\frac{1}{2}) + \sqrt{(j+\frac{1}{2})^2 - (Z\alpha)^2}
      }
    \right)^2
    \right]^{-\frac{1}{2}}
\end{align}

\begin{align}
  E & = - \frac{mc^2}{2n^2} (Z\alpha)^2 \:\:\: \times \nonumber \\
  & \left[ 1 + \frac{(Z\alpha)^2}{n}
  \left( \frac{1}{j+\frac{1}{2}} - \frac{3}{4n}
  \right)
  +O((Z\alpha)^4) \right]
\end{align}


% Reference: Eq. 8.1.18' of SAQM
\begin{align}
  E_{n,\ell} & = mc^2 \:\:\: \times \nonumber \\
    & \left[1 +
    \left(
    \frac{Z\alpha}{
      n - (\ell+\frac{1}{2}) + \sqrt{(\ell+\frac{1}{2})^2 - (Z\alpha)^2}
      }
    \right)^2
    \right]^{-\frac{1}{2}}
\end{align}

\begin{align}
  E & = - \frac{mc^2}{2n^2} (Z\alpha)^2 \:\:\: \times \nonumber \\
  & \left[ 1 + \frac{(Z\alpha)^2}{n}
  \left( \frac{1}{\ell+\frac{1}{2}} - \frac{3}{4n}
  \right)
  +O((Z\alpha)^4) \right]
\end{align}

{\bf WKB}

\begin{equation}
\psi_{\pm} (x) \propto \frac{1}{\sqrt{p(x)}}
  \exp(\pm \frac{i}{\hbar} \int^{x} dx' \: p(x'))
\end{equation}
where
\begin{equation}
p(x) = \sqrt{2 m (E-V(x))}
\end{equation}

\begin{equation}
\frac{1}{\hbar} \int_0^{x_+(E)} dx \: p(x) =
\left(n + \frac{3}{4} \right) \pi
\end{equation}
and
\begin{equation}
\frac{1}{\hbar} \int_{x_-(E)}^{x_+(E)} dx \: p(x) =
\left(n + \frac{1}{2} \right) \pi
\end{equation}
with $n=0,1,2,\ldots$.

{\bf Variational}

\begin{equation}
[ E ] = \frac{\braket{\psi|H|\psi}}{\braket{\psi|\psi}}
\end{equation}

{\bf Some integrals}

\begin{equation}
\int_{-\infty}^{\infty} dx \: e^{-kx^2} = \sqrt{\pi/k}
\end{equation}

\begin{equation} % 18.75 of SHMT5
\int_{-\infty}^{\infty} e^{-\left(a x^2+b x+c\right)} d x=\sqrt{\frac{\pi}{a}} e^{\left(b^2-4 a c\right) / 4 a}
\end{equation}

\begin{equation}
\int_{-\infty}^{\infty} dx \: x^2 e^{-kx^2} = \frac{\sqrt{\pi}}{2 k^{3/2}}
\end{equation}

\begin{equation}
\int_{-\infty}^{\infty} dx \: e^{ix(k''-k')} = 2 \pi \delta (k''-k')
\end{equation}

\begin{equation}
\int_{-\infty}^{\infty} dx \: \exp (\alpha i x^2) = \sqrt{\frac{\pi i}{\alpha}}
\label{eq:simp_int}
\end{equation}

\begin{equation}
\int_{-\infty}^{\infty} dx \; x^2 \exp (\alpha i x^2)
  = \frac{\sqrt{\pi}}{2} \left( \frac{i}{\alpha} \right)^{3/2}
\label{eq:2nd_simp_int}
\end{equation}
{\small ($^{*}$oscillates around this value)}.

\begin{equation}
\int_{0}^{\pi} \sin^{2}{\left (x \right )}\, dx  =  \frac{\pi}{2}
\end{equation}

\begin{equation}
\int_{0}^{2 \pi} \sin^{2}{\left (x \right )}\, dx  =  \pi
\end{equation}

\begin{equation}
\int_{0}^{\frac{\pi}{2}} \sin{\left (x \right )} \sin{\left (2 x \right )}\, dx  =  \frac{2}{3}
\end{equation}

{\bf Legendre Polynomials}

\begin{equation}
P_0(x) = 1
\end{equation}
\begin{equation}
P_1(x) = x
\end{equation}
\begin{equation}
P_2(x) = \frac{1}{2} (3x^2-1)
\end{equation}

{\bf Spherical coordinates}

\begin{eqnarray}
z/r & = & \cos \theta \\
x/r & = & \sin \theta \cos \phi \\
y/r & = & \sin \theta \sin \phi
\end{eqnarray}

\end{multicols}

\pagebreak

Vector identities:

\[
\mathbf { A } \cdot ( \mathbf { B } \times \mathbf { C } ) = \mathbf { B } \cdot ( \mathbf { C } \times \mathbf { A } ) = \mathbf { C } \cdot ( \mathbf { A } \times \mathbf { B } )
\]


\[
\mathbf { A } \times ( \mathbf { B } \times \mathbf { C } ) = \mathbf { B } ( \mathbf { A } \cdot \mathbf { C } ) - \mathbf { C } ( \mathbf { A } \cdot \mathbf { B } )
\]


\[
\boldsymbol { \nabla } ( f g ) = f ( \boldsymbol { \nabla } g ) + g ( \boldsymbol { \nabla } f )
\]


\[
\nabla ( \mathbf { A } \cdot \mathbf { B } ) = \mathbf { A } \times ( \nabla \times \mathbf { B } ) + \mathbf { B } \times ( \nabla \times \mathbf { A } ) + ( \mathbf { A } \cdot \nabla ) \mathbf { B } + ( \mathbf { B } \cdot \nabla ) \mathbf { A }
\]


\[
\boldsymbol { \nabla } \cdot ( f \mathbf { A } ) = f ( \mathbf { \nabla } \cdot \mathbf { A } ) + \mathbf { A } \cdot ( \nabla f )
\]


\[
\nabla \cdot ( \mathbf { A } \times \mathbf { B } ) = \mathbf { B } \cdot ( \mathbf { \nabla } \times \mathbf { A } ) - \mathbf { A } \cdot ( \nabla \times \mathbf { B } )
\]


\[
\boldsymbol { \nabla } \times ( f \mathbf { A } ) = f ( \mathbf { \nabla } \times \mathbf { A } ) - \mathbf { A } \times ( \nabla f )
\]


\[
\nabla \times ( \mathbf { A } \times \mathbf { B } ) = ( \mathbf { B } \cdot \mathbf { \nabla } ) \mathbf { A } - ( \mathbf { A } \cdot \mathbf { \nabla } ) \mathbf { B } + \mathbf { A } ( \mathbf { \nabla } \cdot \mathbf { B } ) - \mathbf { B } ( \mathbf { \nabla } \cdot \mathbf { A } )
\]


\[
\nabla \cdot ( \nabla \times \mathbf { A } ) = 0
\]


\[
\nabla \times ( \nabla f ) = 0
\]


\[
\nabla \times ( \nabla \times \mathbf { A } ) = \nabla ( \nabla \cdot \mathbf { A } ) - \nabla ^ { 2 } \mathbf { A }
\]

\newpage

{\bf Spherical coordinates}

\[
d \mathbf { l } = d r \hat { \mathbf { r } } + r d \theta \hat { \boldsymbol { \theta } } + r \sin \theta d \phi \hat { \boldsymbol { \phi } } ; \quad d \tau = r ^ { 2 } \sin \theta d r d \theta d \boldsymbol { \phi }
\]


\[
\nabla t = \frac { \partial t } { \partial r } \hat { \mathbf { r } } + \frac { 1 } { r } \frac { \partial t } { \partial \theta } \hat { \boldsymbol { \theta } } + \frac { 1 } { r \sin \theta } \frac { \partial t } { \partial \phi } \hat { \boldsymbol { \phi } }
\]



\[
\nabla \cdot \mathbf { v } = \frac { 1 } { r ^ { 2 } } \frac { \partial } { \partial r } \left( r ^ { 2 } v _ { r } \right) + \frac { 1 } { r \sin \theta } \frac { \partial } { \partial \theta } \left( \sin \theta v _ { \theta } \right) + \frac { 1 } { r \sin \theta } \frac { \partial v _ { \phi } } { \partial \phi }
\]


\[
\begin{array} { r l } { \nabla \times \mathbf { v } = } & { \frac { 1 } { r \sin \theta } \left[ \frac { \partial } { \partial \theta } \left( \sin \theta v _ { \phi } \right) - \frac { \partial v _ { \theta } } { \partial \phi } \right] \hat { \mathbf { r } } } \\ { } & { + \frac { 1 } { r } \left[ \frac { 1 } { \sin \theta } \frac { \partial v _ { r } } { \partial \phi } - \frac { \partial } { \partial r } \left( r v _ { \phi } \right) \right] \hat { \boldsymbol { \theta } } + \frac { 1 } { r } \left[ \frac { \partial } { \partial r } \left( r v _ { \theta } \right) - \frac { \partial v _ { r } } { \partial \theta } \right] \hat { \boldsymbol { \phi } } } \end{array}
\]


\[
\nabla ^ { 2 } t = \frac { 1 } { r ^ { 2 } } \frac { \partial } { \partial r } \left( r ^ { 2 } \frac { \partial t } { \partial r } \right) + \frac { 1 } { r ^ { 2 } \sin \theta } \frac { \partial } { \partial \theta } \left( \sin \theta \frac { \partial t } { \partial \theta } \right) + \frac { 1 } { r ^ { 2 } \sin ^ { 2 } \theta } \frac { \partial ^ { 2 } t } { \partial \phi ^ { 2 } }
\]



{\bf Cylindrical coordinates}

\[
d \mathbf { l } = d s \hat { \mathbf { s } } + s d \phi \hat { \boldsymbol { \phi } } + d z \hat { \mathbf { z } } ; \quad d \tau = s d s d \phi d z
\]


\[
\nabla t = \frac { \partial t } { \partial s } \hat { \mathbf { s } } + \frac { 1 } { s } \frac { \partial t } { \partial \phi } \hat { \boldsymbol { \phi } } + \frac { \partial t } { \partial z } \mathbf { \hat { z } }
\]


\[
\nabla \cdot \mathbf { v } = \frac { 1 } { s } \frac { \partial } { \partial s } \left( s v _ { s } \right) + \frac { 1 } { s } \frac { \partial v _ { \phi } } { \partial \phi } + \frac { \partial v _ { z } } { \partial z }
\]


\[
\nabla \times \mathbf { v } = \left[ \frac { 1 } { s } \frac { \partial v _ { z } } { \partial \phi } - \frac { \partial v _ { \phi } } { \partial z } \right] \hat { \mathbf { s } } + \left[ \frac { \partial v _ { s } } { \partial z } - \frac { \partial v _ { z } } { \partial s } \right] \hat { \boldsymbol { \phi } } + \frac { 1 } { s } \left[ \frac { \partial } { \partial s } \left( s v _ { \phi } \right) - \frac { \partial v _ { s } } { \partial \phi } \right] \hat { \mathbf { z } }
\]


\[
\nabla ^ { 2 } t = \frac { 1 } { s } \frac { \partial } { \partial s } \left( s \frac { \partial t } { \partial s } \right) + \frac { 1 } { s ^ { 2 } } \frac { \partial ^ { 2 } t } { \partial \phi ^ { 2 } } + \frac { \partial ^ { 2 } t } { \partial z ^ { 2 } }
\]


\pagebreak


{\large \centerline{Physical constants and conversion factors}}
% from: http://johanw.home.xs4all.nl/contents.html
\def\npar{\par\medskip}
%\def\vec#1{\mbox{\boldmath$#1$\unboldmath}} % Uncomment this when you don't like the arrows
\def\vvec#1{\vec{#1}\,}
\def\rr{\vec{r}}
\def\vv{\vec{v}}
\def\aaa{\vec{a}}
\def\e#1{\vec{e}_{\rm #1}}
\def\ee#1{\vec{e}_{#1}}
\def\Q#1#2{\frac{\partial #1}{\partial #2}}
\def\QQ#1#2{\frac{\partial^2 #1}{\partial #2^2}}
\def\Qc#1#2#3{\left(\frac{\partial #1}{\partial #2}\right)_{#3}}
\def\LL{{\cal L}}
\def\RR{I\hspace{-1mm}R}
\def\NN{I\hspace{-1mm}N}
\def\TT{\mbox{\sfd T}}
\def\DD{\mbox{\sfd D}}
\def\half{\mbox{$\frac{1}{2}$}}
\def\kwart{\mbox{$\frac{1}{4}$}}
\def\av#1{\left\langle #1 \right\rangle}
\def\oiint{\int\hspace{-2ex}\int\hspace{-3ex}\bigcirc~}
\def\iint{\int\hspace{-1.5ex}\int}
\def\iiint{\int\hspace{-1.5ex}\int\hspace{-1.5ex}\int}
\def\dd{d\hspace{-1ex}\rule[1.25ex]{2mm}{0.4pt}}
\def\lrarrow{~\lower.2ex\hbox{$\rightarrow$}\kern-2.4ex\raise.7ex\hbox{$\leftarrow$}~}
\def\rlarrow{~\lower.2ex\hbox{$\leftarrow$}\kern-2.3ex\raise.7ex\hbox{$\rightarrow$}~}
\def\ne{n_{\rm e}}
\def\ni{n_{\rm i}}
\def\no{n_{\rm 0}}
\def\me{m_{\rm e}}
\def\mi{m_{\rm i}}
\def\Te{T_{\rm e}}
\def\Ti{T_{\rm i}}

\def\labelenumii{\Roman{enumii}.}
\def\theenumii{\Roman{enumii}}
\def\p@enumii{\theenumi}
\begin{center}
\begin{tabular}{||l|lll||}
\hline
{\bf Name}&{\bf Symbol}&{\bf Value}&{\bf Unit}\\
\hline
\hline
Elementary charge            &$e$&$1.60217733\cdot10^{-19}$&C\rule{0pt}{13pt}\\
Fine-structure constant      &$\alpha=e^2/2hc\varepsilon_0$&$\approx1/137$&\\
Speed of light in vacuum     &$c$&$2.99792458\cdot10^8$&m/s (def)\\
Permittivity of the vacuum   &$\varepsilon_0$&$8.854187\cdot10^{-12}$&F/m\\
Permeability of the vacuum   &$\mu_0$&$4\pi\cdot10^{-7}$&H/m\\
$(4\pi\varepsilon_0)^{-1}$   &&$8.9876\cdot10^9$&Nm$^2$C$^{-2}$\\
Planck's constant            &$h$&$6.6260755\cdot10^{-34}$&Js\rule{0pt}{13pt}\\
                             &$\hbar=h/2\pi$&$1.0545727\cdot10^{-34}$&Js\\
Electron mass                &$m_{\rm e}$&$9.1093897\cdot10^{-31}$&kg\rule{0pt}{13pt}\\
Proton mass                  &$m_{\rm p}$&$1.6726231\cdot10^{-27}$&kg\\
Elementary mass unit         &$m_{\rm u}=\frac{1}{12}m(^{12}_{~6}$C)&$1.6605656\cdot10^{-27}$&kg\\
barn & b & 10$^{-28}$ & m$^2$ \\
& $\hbar c$ & $\approx 197$ & \SI{}{MeV\cdot fm}\\
& fm & \SI{e-15}{} & \SI{}{m} \\
\hline
\end{tabular}
\end{center}

% Include Particle Data Group Clebsch-Gordan table:
% (downloaded from: https://goo.gl/aBUKnk )
  \includepdf[pages={1},pagecommand={
    \begin{tikzpicture}[remember picture, overlay]
      \node[anchor=west] at (0, 1.3)
           {\bf From the {\it Review of Particle Physics}
                         {\footnotesize \url{http://pdg.lbl.gov}}};
    \end{tikzpicture}}]{artifacts/rpp2018-rev-clebsch-gordan-coefs.pdf}
\end{document}
